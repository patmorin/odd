\documentclass{patmorin}
\listfiles
\usepackage{pat}
\usepackage{paralist}
\usepackage{dsfont}  % for \mathds{A}
\usepackage[utf8x]{inputenc}
\usepackage{skull}
\usepackage{paralist}
\usepackage{graphicx}
\usepackage[noend]{algorithmic}

\usepackage[normalem]{ulem}
\usepackage{cancel}
\usepackage{enumitem}

\usepackage{todonotes}

\usepackage[longnamesfirst,numbers,sort&compress]{natbib}

\newcommand{\Rho}{\mathrm{P}}

% \newcommand{\harpoon}{\overset{\rightharpoonup}}
\newcommand{\qa}{\overset{\rightharpoonup}{\varphi}}
\newcommand{\qb}{\overset{\rightharpoondown}{\varphi}}
\newcommand{\qap}{\overset{\rightharpoonup}{\sigma}}
\newcommand{\qbp}{\overset{\rightharpoondown}{\sigma}}

\usepackage[mathlines]{lineno}
\setlength{\linenumbersep}{2em}
% \linenumbers
% \rightlinenumbers
% \linenumbers
\newcommand*\patchAmsMathEnvironmentForLineno[1]{%
 \expandafter\let\csname old#1\expandafter\endcsname\csname #1\endcsname
 \expandafter\let\csname oldend#1\expandafter\endcsname\csname end#1\endcsname
 \renewenvironment{#1}%
    {\linenomath\csname old#1\endcsname}%
    {\csname oldend#1\endcsname\endlinenomath}}%
\newcommand*\patchBothAmsMathEnvironmentsForLineno[1]{%
 \patchAmsMathEnvironmentForLineno{#1}%
 \patchAmsMathEnvironmentForLineno{#1*}}%
\AtBeginDocument{%
\patchBothAmsMathEnvironmentsForLineno{equation}%
\patchBothAmsMathEnvironmentsForLineno{align}%
\patchBothAmsMathEnvironmentsForLineno{flalign}%
\patchBothAmsMathEnvironmentsForLineno{alignat}%
\patchBothAmsMathEnvironmentsForLineno{gather}%
\patchBothAmsMathEnvironmentsForLineno{multline}%
}


\newcommand{\coloured}[2]{{\color{#1}{#2}}}
\newenvironment{colourblock}[1]{\color{#1}}{}

\newcommand{\condref}[1]{(C\ref{#1})}

% Taken from
% https://tex.stackexchange.com/questions/42726/align-but-show-one-equation-number-at-the-end
\newcommand\numberthis{\addtocounter{equation}{1}\tag{\theequation}}


\setlength{\parskip}{1ex}


\DeclareMathOperator{\diam}{diam}
\DeclareMathOperator{\tw}{tw}
\DeclareMathOperator{\lca}{lca}

\DeclareMathOperator{\x}{x}
\DeclareMathOperator{\height}{height}
\DeclareMathOperator{\depth}{depth}
\DeclareMathOperator{\dist}{dist}
\DeclareMathOperator{\sh}{cbt}
\DeclareMathOperator{\cbt}{cbt}
\DeclareMathOperator{\sgn}{sgn}
\DeclareMathOperator{\dc}{dc}

\title{\MakeUppercase{Odd Colourings of Graph Products}\thanks{This research was partly funded by NSERC.}}
\author{%
  Vida Dujmovic\thanks{Department of Computer Science and Electrical Engineering, University of Ottawa}\qquad
  Pat Morin\thanks{School of Computer Science, Carleton University}
}

\date{}


\newcommand{\colored}[2]{{\color{#1}#2}}

\usepackage{tabularx}

\DeclareMathOperator{\odd}{odd}

\begin{document}

% \begin{titlepage}
\maketitle

\begin{abstract}
  Todo
\end{abstract}
% \end{titlepage}

% \pagenumbering{roman}
% \tableofcontents
%
% \newpage
% \pagenumbering{arabic}

\section{Introduction}

Let $G$ be a graph.  A (not necessarily proper)\footnote{$varphi$ is a proper colouring of $G$ if $vw\in E(G)$ implies that $\varphi(v)\neq\varphi(w)$.} vertex colouring $\varphi:V(G)\to\N$ is \emph{odd} if the neighbourhood of each vertex of $G$ contains a colour that occurs an odd number of times.  More precisely, if $N_G(v):=\{w\in V(G):vw\in E(G)\}$ denotes the neighbourhood of a vertex $v$ in $G$, then $\varphi$ is an odd colouring of $G$ if and only if, for each $v\in V(G)$ there exists a colour $\alpha$ such that $|\{w\in N_G(v): \varphi(w)=\alpha\}|$ is odd.

Odd colourings were recently introduced by \citet{petrusevski.skrekovski:colorings}, who showed that every connected planar graph $G$ with $n\ge 2$ vertices has a proper odd colouring using at most $9$ colours, and conjectured that $5$ colours always suffice.  \citet{caro.petrusevski.ea:remarks} showed that $5$ colours always suffice for connected outerplanar graphs.

A minor-closed family of graphs $\mathcal{G}$ is \emph{$d$-degenerate} if every graph in $G$ contains a vertex of degree at most $d$.
% (Note that any proper minor-closed family of graphs is $d$-degenerate for some fixed $d$.) A vertex of degree zero is called an \emph{isolated} vertex.
\citet{cranston.lafferty.ea:note} proved that any connected graph from a $d$-degenerate minor-closed family of graphs has a proper odd colouring using at most $2d+1$ colours.  This result includes outerplanar graphs (width $d=2$); planar graphs (with $d=5$); graphs embeddable on surfaces of Euler genus $g$ (with $d=X$); and graphs of treewidth at most $t$ (with $d=t$).

\citet{cranston.lafferty.ea:note} also consider $1$-planar graphs, which do not form a minor-closed family, and show that any $1$-planar graph has a proper odd colouring using at most $31$ colours.  They ask if this can be extended to $k$-planar graphs for general $k>1$.  Our main result, \cref{new_result} below, is a more general results for graphs having \emph{product structure} and implies that, for any fixed $k$, any connected $k$-planar graph has a proper odd colouring using $O(k^5)$ colours.
%
% \begin{thm}\label{old_result}
%   Let $\mathcal{H}$ be a $d$-degenerate minor closed family of graphs and let $H$ be a member of $\mathcal{H}$ with no isolated vertices.  Then $H$ has a proper odd colouring $\varphi:V(H)\to\{1,\ldots,2d+1\}$.
% \end{thm}

\begin{thm}\label{new_result}
  Let $H$ be a $t$-tree, let $P$ be a path, and let $G$ be a subgraph of $H\boxtimes P$ with no isolated vertices. Then $G$ has a proper odd colouring $\varphi:V(H)\to\{1,\ldots,6t+3\}$.
\end{thm}

\begin{proof}
  For minor technical reasons it will be helpful to relax the definition of odd colouring to allow isolated vertices.  To do this, we say that a colouring $\varphi$ is \emph{odd} if, for each \emph{non-isolated} vertex $v$ of $G$, there exists $\alpha$ such that $|\{w\in N_G(v):\varphi(w)=\alpha\}|$ is odd.

  Let $y_1,\ldots,y_h$ be the vertices of $P$, in order.  Let $x_1,\ldots,x_r$ be the vertices of $H$ ordered so that $x_1,\ldots,x_{t}$ is a clique and, for each $i\in\{t+1,\ldots,r\}$, $C_i:=N_G(x_i)\cap\{x_1,\ldots,x_{i-1}\}$ is a clique of size $t$.  The proof is by induction on the number of vertices of $G$ (which may have isolated vertices) and will prove the existence of a proper odd colouring $\varphi:V(G)\to\{1,\ldots,6t+3\}$ with the following additional property:
  \begin{compactitem}[(*)]
    \item For each $i,j\in\{1,\ldots,r\}\times \{2,\ldots,h-1\}$, define
    \[
      C_{(x_i,y_j)}:=C_{i,j}:=V(G)\cap ((\{x_i\}\cup C_i)\times\{y_{j-1}, y_{j},y_{j+1}\})
      \enspace .
    \]
    We will produce a colouring in which all vertices in $C_{i,j}$ receive distinct colours, for each $i,j\in\{1,\ldots,r\}\times\{2,\ldots,h-1\}$.
  \end{compactitem}
  Note that, for any edge $vw$ of $H\boxtimes P$, $v \in C_v$ or $w\in C_w$.  Therefore, Property~(P1) implies that the colouring $\varphi$ is a proper colouring of $H\boxtimes P$ and therefore also a proper colouring of $G$.


  We first dispense with some easy cases.  If $H$ has $t+1$ vertices, then we can easily find a distance-$2$ colouring $\varphi$ of $H\boxtimes P$, using $3t+3\le 6t+3$ colours. In particular, we can begin with any proper $(t+1)$-colouring $\rho:V(H)\to\{0,\ldots,t\}$ of $H$ and define $\varphi(x_i,y_j):=1+((t+1)j + \rho(x_i))\bmod (3t+3)$.  This colouring satifies (*) and, for any $v\in V(H\boxtimes P)$ each vertex in $N_{H\boxtimes P}(v)$ receives a distinct colour.  Therefore, for any non-isolated vertex $v\in V(G)$, $N_G(v)$ contains a vertex whose colour is unique, so $\rho$ is an odd colouring of $G$.

  % If $r=1$, then the proof is easy since, in this case $H\boxtimes P$ is a path, $G$ is a collection of paths and $G$ can easily be coloured with $3=6\cdot 0+3\le 3\cdot d+3$ colours.
  The base case, in which $G$ has no vertices, is trivial. Therefore, we may assume that $G$ has at least one vertex.  Let $(i,j)\in\{1,\ldots,r\}\times\{1,\ldots,h\}$ be the lexicographically largest pair such that $(x_i,y_j)\in V(G)$.  Since $G$ has at least one vertex, $(i,j)$ is well-defined.  Let $v:=(x_i,y_j)$.

  Consider the graph $G':= G-\{v\}$.  By the inductive hypothesis, we can find an odd colouring $\varphi$ of $G'$ that satisfies (*).  We will now extend $\varphi$ to a colouring of $G$ by first listing colours that we may not choose for $\varphi(v)$:
  \begin{compactitem}
    \item To guarantee that $\varphi$ has Property~(*), we avoid colours in the set $X:= \{\varphi(w): w\in C_v\setminus\{v\}\}$. By definition, if we choose $\varphi(v)\not\in X$ then the resulting colouring will satisfy Property~(*).  Furthermore, unless $v$ is an isolated vertex of $G'$, it also ensures that $N_{G}(v)$ has a colour that occurs exactly once (because every vertex in $C_v\supseteq N_{G}(v)$ is assigned a unique colour).

    \item To guarantee that $\varphi$ is odd, we consider each vertex $w\in N_{G}(v)$.  If there is exactly one colour $\alpha\in\{1,\ldots,6d+3\}$ that occurs an odd number of times in $N_{G'}(w)$, then we define $Y_{w} := \{\alpha\}$; otherwise we define $Y_{w}:=\emptyset$. Now let $Y:=\bigcup_{w\in N_{G}(v)} Y_{w}$.
    If we choose $\varphi(v)\not\in Y$ then , for each $w\in N_{G}(v)$ the following holds:
    \begin{compactitem}
      \item If $Y_{w}=\{\alpha\}$ then $\varphi(v)\neq\alpha$, so the colour $\alpha$ appears an odd number of times in $N_{G}(w)$.
      \item If $Y_{w}=\emptyset$ then either:
      \begin{compactenum}
        \item No colour appears an odd number of times in $N_{G'}(w)$ in which case the colour $\varphi(v)$ appears an odd number of times in $N_{G}(w)$.
        \item At least two colours $\alpha$ and $\beta$ each appear an odd number of times in $N_{G'}(w)$ in which case each colour in $\{\alpha,\beta\}\setminus\{\varphi(v)\}$ appears an odd number of times in $N_{G}(w)$.  In particular, there is at least one colour that appears an odd number of times in $N_G(w)$.
      \end{compactenum}
    \end{compactitem}
  \end{compactitem}
  Therefore, by choosing $\varphi(v)\not\in X\cup Y$ we obtain an odd colouring of $G$ that satisifies (*).  Each of the sets $X$ and $Y$ has size at most $3d+1$, so $|X\cup Y|\le 6d+2$.  Therefore we can always find a value $\varphi(v)\in \{1,\ldots,6t+3\}\setminus(X\cup Y)$ that completes the colouring of $G$.
\end{proof}

%
% \section{Preliminaries}
%
%
%
%
%
%
% We will make use of the following trivial observation:
%
% \begin{obs}\label{odd_sum}
%   Let $r_1,\ldots,r_k$ be integers and let $r:=\sum_{i=1}^k r_i$.  If $r$ is odd then at least one of $r_1,\ldots,r_k$ is odd.
% \end{obs}
%
% \cref{odd_sum} is useful because it allows us to make use of product colourings.  We say that a colouring $\varphi$ of $G$ is \emph{$\alpha$-odd at $v$} for $G$ if the colour $\alpha$ appears an odd number of times in $N_G(v)$, i.e., $|\{w\in N_G(v): \varphi(w)=\alpha\}|\equiv 1\pmod 2$.  We say that $\varphi$ is \emph{odd at $v$} if $\varphi$ is $\alpha$-odd at $v$ for some colour $\alpha$.
%
% \begin{obs}\label{product_colouring}
%   Let $G$ be a graph and let $\varphi,\rho$ be two colourings of $G$ such that at least one of $\varphi$ or $\rho$ is odd at $v$.  Then the product colouring $\psi$ defined as $\psi(v):=(\varphi(v),\rho(v))$ is odd at $v$.
% \end{obs}
%
% \begin{proof}
%   Without loss of generality, assume that $\varphi$ is $\alpha$-odd at $v$. In other words, $|\{w\in N_G(v):\varphi(w)=\alpha\}|$ is odd.  Therefore, by \cref{odd_sum}, there exists $\beta$ such that $|\{w\in N_G(v): \psi(w)=(\alpha,\beta)\}|$ is odd.  Therefore $\psi$ is odd at $v$.
% \end{proof}
%
% For a graph $G$ and a set $X$, $G[X]$ is the vertex-induced subgraph with $V(G[X]):=V(G)\cap X$ and $E(G[X]):=\{vw\in E(G):v,w\in X\}$.  For two sets $X$ and $Y$, $G[X,Y]$ is the graph with $V(G[X,Y]):= V(G)\cap (X\cup Y)$ and $E(G):=\{vw\in v\in X, w\in Y\}$.
%
% A \emph{layering} of a graph $G$ is partition of $V(G)$ into a sequence of sets $(L_1,\ldots,L_h)$ such that, for each edge $vw\in E(G)$, $v\in L_i$ and $w\in L_j$ implies that $|i-j|\le 1$. For convenience, we will use the convention that $L_0:=L_{h+1}:=\emptyset$ and we will let $\ell(v)$ be defined so that $v\in L_{\ell(v)}$.
%
% \begin{lem}
%   Let $G$ be a graph, let $L_1,\ldots,L_h$ be a layering of $G$, and let $\varphi$, $\rho$, and $\delta$ be colourings of $G$ such that, for each $i\in\{1,\ldots,h\}$ and each $v\in L_i$,
%   \begin{compactenum}
%     \item $\varphi$ is odd at $v$ for $G[L_i]$;
%     \item $\rho$ is odd at $v$ for $G[L_i,L_{i-1}]$; or
%     \item $\delta$ is odd at $v$ for $G[L_i,L_{i+1}]$.
%   \end{compactenum}
%   Then the product colouring $\psi$ defined as $\psi(v):=(\varphi(v),\rho(v),\delta(v),\ell(v)\bmod 3)$ is an odd colouring of $G$.
% \end{lem}
%
% \begin{proof}
%   If $\varphi$ is $\alpha$-odd at $v$ for $G[L_i]$, then
%   \[
%        |\{w\in N_G(v):\varphi(w)=\alpha,\, w\in L_{i}\}|=|\{w\in N_G(v):\varphi(w)=\alpha,\, \ell(v)=i\bmod 3\}|
%   \]
%   is odd.  Therefore, by \cref{product_colouring}, $\psi$ is odd at $v$.
%
%   Similarly, if $\rho$ is $\beta$-odd at $v$ for $G[L_{i},L_{i-1}]$ then $|\{w\in N_G(v):\rho(w)=\beta,\,\ell(v)=(i-1)\bmod 3\}|$ is odd.  So, by  \cref{product_colouring}, $\psi$ is odd at $v$.
%
%   Again, similarly, if $\delta$ is $\gamma$-odd at $v$ for $G[L_{i},L_{i+1}]$ then $|w\in N_G(v):\delta(v)=\gamma,\ell(v)=(i+1)\bmod 3\}|$ is odd.  So, by \cref{product_colouring}, $\psi$ is odd at $v$.
%
%   Therefore, $\psi$ is odd
% \end{proof}
%
%
%
%
%
%
%
%
%
%

\bibliographystyle{plainurlnat}
\bibliography{odd}


\end{document}
