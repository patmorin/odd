\documentclass{patmorin}
\listfiles
\usepackage{pat}
\usepackage{paralist}
\usepackage{dsfont}  % for \mathds{A}
\usepackage[utf8x]{inputenc}
\usepackage{skull}
\usepackage{paralist}
\usepackage{graphicx}
\usepackage[noend]{algorithmic}

\usepackage[normalem]{ulem}
\usepackage{cancel}
\usepackage{enumitem}

\usepackage{pifont}
\usepackage{todonotes}

\usepackage{relsize}

\usepackage[longnamesfirst,numbers,sort&compress]{natbib}


\usepackage[mathlines]{lineno}
\setlength{\linenumbersep}{2em}
% \linenumbers
% \rightlinenumbers
% \linenumbers
\newcommand*\patchAmsMathEnvironmentForLineno[1]{%
 \expandafter\let\csname old#1\expandafter\endcsname\csname #1\endcsname
 \expandafter\let\csname oldend#1\expandafter\endcsname\csname end#1\endcsname
 \renewenvironment{#1}%
    {\linenomath\csname old#1\endcsname}%
    {\csname oldend#1\endcsname\endlinenomath}}%
\newcommand*\patchBothAmsMathEnvironmentsForLineno[1]{%
 \patchAmsMathEnvironmentForLineno{#1}%
 \patchAmsMathEnvironmentForLineno{#1*}}%
\AtBeginDocument{%
\patchBothAmsMathEnvironmentsForLineno{equation}%
\patchBothAmsMathEnvironmentsForLineno{align}%
\patchBothAmsMathEnvironmentsForLineno{flalign}%
\patchBothAmsMathEnvironmentsForLineno{alignat}%
\patchBothAmsMathEnvironmentsForLineno{gather}%
\patchBothAmsMathEnvironmentsForLineno{multline}%
}



% Taken from
% https://tex.stackexchange.com/questions/42726/align-but-show-one-equation-number-at-the-end
\newcommand\numberthis{\addtocounter{equation}{1}\tag{\theequation}}


\setlength{\parskip}{1ex}


\DeclareMathOperator{\tw}{tw}

\DeclareMathOperator{\odd}{\chi_o}
\newcommand{\oddc}{\chi_{o\ell}}

\title{\MakeUppercase{Odd Colourings of Graph Products and Layered Decompositions}\thanks{This research was partly funded by NSERC.}}
\author{%
  Vida Dujmović\thanks{Department of Computer Science and Electrical Engineering, University of Ottawa}\qquad
  Chun-Hung Liu\thanks{Department of Mathematics, Texas A\&M University}\qquad
  Pat Morin\thanks{School of Computer Science, Carleton University}\qquad
  Saeed Odak\footnotemark[2]
}

\date{}

\newcommand{\colored}[2]{{\color{#1}#2}}

\usepackage{tabularx}


\begin{document}

% \begin{titlepage}
\maketitle

\begin{abstract}
  The odd chromatic number is a new graph parameter introduced by \citet{petrusevski.skrekovski:colorings}.  In this note, we show that graphs of bounded layered treewidth and graphs with so called product structure have bounded odd chromatic number and bounded odd choice number, respectively. By known results on the layered treewidth and product structure of $k$-planar graphs, this implies that $k$-planar graphs have bounded odd chromatic number and choice number, which answers a question of \citet{cranston.lafferty.ea:note}.
\end{abstract}
% \end{titlepage}

% \pagenumbering{roman}
% \tableofcontents
%
% \newpage
% \pagenumbering{arabic}

\section{Introduction}

Let $G$ be a graph.  A (not necessarily proper)\footnote{$\varphi$ is a proper colouring of $G$ if $vw\in E(G)$ implies that $\varphi(v)\neq\varphi(w)$.} vertex colouring $\varphi:V(G)\to\N$ is \emph{odd} if, for each non-isolated vertex $v$, there exists a colour that appears at an odd number of $v$'s neighbours. More precisely, if $N_G(v):=\{w\in V(G):vw\in E(G)\}$ denotes the neighbourhood of a vertex $v$ in $G$, then $\varphi$ is an odd colouring of $G$ if and only if, for each $v\in V(G)$ with $|N_G(v)|>0$, there exists a colour $\alpha$ such that $|\{w\in N_G(v): \varphi(w)=\alpha\}|$ is odd.  For any graph $G$, the \emph{odd chromatic number} $\odd(G)$ of $G$ is the minimum number of colours used\footnote{We say that a colouring $\varphi$ of $G$ \emph{uses $c$ colours} if $c=|\{\varphi(v):v\in V(G)\}|$.} by any proper odd colouring of $G$.

\subsection{Previous Work}

Odd colourings were recently introduced by \citet{petrusevski.skrekovski:colorings}, who showed that $\odd(G)\le 9$ for every planar graph $G$ and conjectured that $\odd(G)\le 5$ for every planar graph $G$. \citet{caro.petrusevski.ea:remarks} showed that $\odd(G)\le 5$ for every outerplanar graph $G$ and that $\odd(G)\le 8$ for some some special cases of planar graphs $G$.    Building on the work of \citet{caro.petrusevski.ea:remarks}, \citet{petr.portier:odd} showed that $\odd(G)\le 8$ for every planar graph $G$.

A minor-closed family of graphs\footnote{A graph $M$ is a \textit{minor} of a graph $G$ if a graph isomorphic to $M$ can be obtained from a subgraph of $G$ by contracting edges. A class $\mathcal{G}$ of graphs is \emph{minor-closed} if for every graph $G\in\mathcal{G}$, every minor of $G$ is in $\mathcal{G}$.} $\mathcal{G}$ is \emph{$d$-degenerate} if every graph in $\mathcal{G}$ contains a vertex of degree at most $d$.
% (Note that any proper minor-closed family of graphs is $d$-degenerate for some fixed $d$.) A vertex of degree zero is called an \emph{isolated} vertex.
\citet{cranston.lafferty.ea:note} proved that $\odd(G)\le 2d+1$ for any graph $G$ that belongs to a $d$-degenerate minor-closed family of graphs.  This result, which has a short and elegant proof, includes outerplanar graphs and, more generally, partial $2$-trees (with $d=2$); planar graphs (with $d=5$); graphs embeddable on surfaces of Euler genus $g$; and graphs of treewidth at most $t$ (with $d=t$).

\citet{cranston.lafferty.ea:note} also consider $1$-planar graphs,\footnote{A graph is \emph{$k$-planar} if it has an embedding in the plane in which no edge contains a vertex other than its endpoints and each edge is involved in at most $k$ crossings with other edges.} which do not form a minor-closed family, and show that any $1$-planar graph $G$ has $\odd(G)\le 23$. Their proof is a discharging argument which relies heavily on structure that is specific to $1$-planar graphs.  They state that ``It seems non-trivial to prove a more general result for $k$-planar graphs for all $k\ge 1$.''

% Their proof, which uses discharging, does not obviously generalize establish a bound on $\odd_c(G)$.\todo{Is this true?} The same authors ask if a similar result can be shown for $k$-planar graphs for general $k>1$.

\todo[inline]{Check occurrences of ``proper odd everywhere''}

\todo[inline]{Check occurrences of ``respects $\ell$'' everywhere}


\subsection{New Results}

In this paper, we establish an upper bound for the odd chromatic number of $k$-planar graphs for any $k$ by proving more general results on graphs that have
certain kinds of layered decompositions, including graphs of layered treewidth at most $t$ and graphs contained in the strong product of a graph of treewidth $t$ and a path.  (Treewidth, layered treewidth and strong products are formally defined in \cref{layered_treewidth} and \cref{product_structure}.)
% Each of these graph classes have been used recently to make progress and resolve a number of longstanding open problems on both minor-closed and non-minor closed graph classes \cite{X,X,X,X,X,}.
In each case we establish bounds with a linear dependence on $t$.

For graphs of bounded layered treewidth we establish bounds on the odd chromatic number:

\begin{thm}\label{layered_treewidth_result}
  If $G$ is a graph with layered treewidth at most $t$ then $\odd(G)\le 16t-2$.
\end{thm}

% \citet[Theorem 10]{dujmovic.eppstein.ea:genus} show that every $k$-planar graph has layered treewidth at most $6(k+1)$.

% More generally,
A graph is \emph{$(g,k)$-planar graph} if it has an embedding on a surface of Euler genus $g$ in which each edge is involved in at most $k$ crossings with other edges.  Thus, a graph is $k$-planar if and only if it is $(0,k)$-planar.  \citet[Theorem~13]{dujmovic.eppstein.ea:genus} show that every $(g,k)$-planar graph has layered treewidth at most $(4g+6)(k+1)$.  Along with \cref{layered_treewidth_result}, this immediately implies the following result, and the special case in which $g=0$ provides the result for $k$-planar graphs suggested by \citet{cranston.lafferty.ea:note}.

% \begin{cor}\label{k_planar}
%   If $G$ is a $k$-planar graph then $\odd(G)\le 96k+94$.\todo{Drop this and only keep the $(g,k)$-planar version?}
% \end{cor}

\begin{cor}\label{gk_planar}
  If $G$ is a $(g,k)$-planar graph then $\odd(G)\le 64g(k+1)+96k+94$.
\end{cor}

Several other non-minor closed families of graphs are known to have bounded layered treewidth, including $(g,d)$-map graphs, $(g,\delta)$-string graphs, and ($2$-dimensional) $r$-nearest neighbour graphs.\footnote{Graphs in these classes are $(g,k)$-planar for some $k$ that depends on $d$, $\delta$, or $r$.  However, it is usually possible to obtain tighter bounds on layered treewidth using direct arguments rather than just apply \cref{gk_planar}.}  \cref{layered_treewidth_result} gives bounds on the odd chromatic number for all graphs in these classes.

Like other colouring problems, odd colouring can also be studied through the lens of list-colouring/choosability.  A \emph{$c$-list assignment} of a graph $G$ is a function $L:V(G)\to\binom{\N}{c}$ that assigns to each vertex $v$ of $G$ a set $L(v)$ of $c$ colours.  We say that a colouring $\varphi$ of $G$ is an $L$-coloring if $\varphi(v)\in L(v)$ for each vertex $v$ of $G$.  The \emph{odd choice number} $\oddc(G)$ of $G$ is the minimum value of $c$ such that, for any $c$-list assignment $L$ of $G$, $G$ has a proper odd $L$-colouring. Clearly $\odd(G)\le \oddc(G)$ for any graph $G$ since, by taking $L(v):=\{1,\ldots,c\}$ for each $v\in V(G)$, any $L$-colouring of $G$  uses at most $c$ colours.  The proof given by \citet{cranston.lafferty.ea:note} applies, without modification to this setting:

\begin{thm}[\citet{cranston.lafferty.ea:note}]\label{degenerate}
  If $G$ belongs to a $d$-degenerate minor-closed family of graphs then $\oddc(G)\le 2d+1$.
\end{thm}

\todo[inline]{
Chun-Hung suggests the following, but I'm not sure we should add an entire section just to repeat someone else's proof.  We could just include the 1-paragraph proof right here(?)

13. Add a new section called "preliminary" or something else before the
current Section 2. In this section, first state Theorem 1 (i.e. the
lemma of Cranston et al) and state the following lemma:

Lemma: Let $d$ be a nonnegative integer and ${\mathcal F}$ a
$d$-degenerate minor-closed family. If $G$ is a graph in ${\mathcal F}$
and $L$ is a $(2d+1)$-list-assignment of $G$, then $G$ has a proper odd
$L$-coloring.

And say that our proof of above lemma is based on the same idea of
identical to the proof in the paper of Cranston et al, but we include
the proof for completeness.

After the proof, state the corollary by using the language of odd choice
number. And say the odd choice number of planar graphs, bounded
treewidth graphs etc is at most something by choosing the correct d.
}

Graphs with product structure are special cases of graphs with bounded layered treewidth. We can improve the bound in \cref{layered_treewidth_result} for those cases, even for the odd choice number.

\begin{thm}\label{product_structure_result}
  If $G$ is a subgraph of $H\boxtimes P$ where $H$ is a graph of treewidth at most $t$ and $P$ is a path then $\oddc(G)\le 8t+4$.
\end{thm}

\citet[Theorem 2]{dujmovic.morin.ea:structure} show that every $k$-planar graph is isomorphic to a subgraph of the strong product of a graph of treewidth $O(k^5)$ and a path, which establishes the following bound on the odd choice number of $k$-planar graphs:

\begin{cor}\label{k_planar_choice}
  If $G$ is a $k$-planar graph then $\oddc(G)\in O(k^5)$.
\end{cor}

A graph $G$ that satisfies the requirements of \cref{product_structure_result} is said to have \emph{row treewidth} at most $t$.  Again, several other non-minor-closed families of graphs are known to have bounded row treewidth, including $k$-gap planar graphs, $(g,d)$-map graphs, $(g,\delta)$-string graphs, ($2$-dimensional) $r$-nearest neighbour graphs, and powers of bounded-degree graphs from proper minor-closed families \cite{dujmovic.morin.ea:structure,hickingbotham.wood:shallow}. \cref{product_structure_result} gives bounds on the odd choice number for all graphs in these classes.
%
%
%
% Our main result, \cref{product_structure_result} below, is a more general results for graphs having \emph{product structure} and implies that any $k$-planar graph has a proper odd colouring using $O(k^5)$ colours.
%
% % \begin{thm}\label{old_result}
% %   Let $\mathcal{H}$ be a $d$-degenerate minor closed family of graphs and let $H$ be a member of $\mathcal{H}$ with no isolated vertices.  Then $H$ has a proper odd colouring $\varphi:V(H)\to\{1,\ldots,2d+1\}$.
% % \end{thm}

In the remainder of this paper we prove \cref{layered_treewidth_result} in \cref{layered_treewidth}, explore a related notion of $(\ell,c)$-good layerings, and then prove \cref{product_structure_result} and some of its generalizations in \cref{product_structure}.

\section{Layered Treewidth (Proof of \cref{layered_treewidth_result})}
\label{layered_treewidth}

We begin with an easy observation that allows us to perform divide-and-conquer:

\begin{obs}\label{partition}
  Let $G$ be a graph and let $\{V_1,\ldots,V_d\}$ be a partition of $V(G)$ such that, for each $i\in\{1,\ldots,d\}$, $G[V_i]$ has no isolated vertices.  Then $\odd(G) \le \sum_{i=1}^d \odd(G[V_i])$.
\end{obs}

\begin{proof}
  Let $C_1,\ldots,C_d$ be pairwise disjoint sets of colours with $|C_i|=\odd(G[V_i])$.  For each $i\in\{1,\ldots,d\}$, let $\varphi_i:V_i\to C_i$ be a proper odd colouring of $G[V_i]$.  For each $i\in\{1,\ldots,d\}$ and each $v\in V_i$, let $\varphi(v):=\varphi_i(v)$.  Clearly $\varphi$ is a colouring of $G$ that uses at most $\sum_{i=1}^d \odd(G[V_i])$ colours, so we only need to verify that $\varphi$ is a proper odd colouring of $G$.
  \begin{compactitem}
    \item To see that $\varphi$ is a proper colouring of $G$, let $vw$ be an edge of $G$ with $v\in V_i$ and $w\in V_j$ for some $i,j\in\{1,\ldots,d\}$. If $i=j$ then $\varphi(v)\neq\varphi(w)$ since $\varphi$ is a proper colouring of $G[V_i]$.  If $i\neq j$, then $\varphi(v)\neq\varphi(w)$ since $\varphi(v)\in C_i$ and $\varphi(v)\in C_j$.

    \item To see that $\varphi$ is an odd colouring of $G$, let $v$ be a vertex in $V_i$ for some $i\in\{1,\ldots,d\}$.   Since $v$ is not an isolated vertex in $G[V_i]$, some colour $\alpha$ appears an odd number, $a$, of times in $N_{G[V_i]}(v)$.  Since $\alpha\not\in C_j$ for any $j\neq i$, $\alpha$ also appears $a$ times in $N_G(v)$. \qedhere
  \end{compactitem}
\end{proof}

A \emph{layering} of a graph $G$ is an ordered partition $\mathcal{L}:=L_1,\ldots,L_h$ of $V(G)$ such that, for each edge $vw$ of $G$, there exists an integer $i$ such that $\{v,w\}\subseteq L_i\cup L_{i+1}$.  For convenience, we will use the convention that $L_i:=\emptyset$ for each $i < 1$ and each $i> h$.  We say that $\mathcal{L}$ is \emph{$(\ell,c)$-good} if, for each integer $i$ and each subgraph $G'$ of $G[L_i\cup\cdots\cup L_{i+\ell-1}]$, $\odd(G')\le c$.


\begin{lem}\label{four_good}
  If a graph $G$ has a $(4,c)$-good layering then $\odd(G)\le 2c$.
\end{lem}

\begin{proof}
  Without loss of generality, we may assume that $G$ has no isolated vertices.
  Let $\mathcal{L}:=L_1,\ldots,L_h$ be a $(4,c)$-good layering of $G$,
  let $X_1:=\bigcup\{L_i:i\bmod 4\in\{0,1\}\}$ and let $X_2:=V(G)\setminus X_1=\bigcup\{L_i:i\bmod 4\in\{2,3\}\}$.  Let $I_2$ be the set of isolated vertices in $G[X_2]$, let $Y_1:=X_1\cup I_2$, and let $I_1$ be the set of isolated vertices in $G[Y_1]$.  Finally, let $V_1:=Y_1\setminus I_1$ and let $V_2:=V(G)\setminus V_1=X_2\cup I_1\setminus I_2$.

  We first claim that, for each $i\in\{1,2\}$, $G_i:=G[V_i]$ has no isolated vertices. For $G_1$ this is obvious, since by definition $G_1$ is the subgraph obtained from $G[Y_1]$ by removing all isolated vertices.  Now consider an arbitrary vertex $v$ of $G_2$.
  \begin{compactenum}
    \item If $v\in X_2\setminus I_2$ then $v\not\in I_2$ so $v$ has a neighbour $w$ in $G[X_2]$.  Then $w$ is not an isolated vertex of $G[X_2]$ so $w\not\in I_2$.  Therefore $w\in V(G_2)\supseteq X_2\setminus I_2$. Therefore $vw$ is an edge of $G_2$ so $v$ is not an isolated vertex of $G_2$.
    % Pat had this:
    % \item If $v\in I_1$ then, because $G$ has no isolated vertices, $v$ has a neighbour $w$ in $G$.  Since $v\in I_1$, $v$ is isolated in $G[Y_1]$ so $w\not\in Y_1= X_1\cup I_2$. Therefore $w\in V(G_2)\supseteq X_2\setminus I_2$.  Therefore $vw$ is an edge of $G_2$ so $v$ is not an isolated vertex of $G_2$.
    % Chun-Hung suggested this:
    \item If $v \in I_1$ then every neighbor of $v$ in $G$ is contained in $V(G) \setminus Y_1 \subseteq V(G_2)$. Since $G$ has no isolated vertices, $v$ has a neighbor in $G_2$. So $v$ is not an isolated vertex in $G_2$.
  \end{compactenum}
  Next we claim that, for each $i\in\{1,2\}$, each connected component of $G_i$ is contained in at most four consecutive layers of $\mathcal{L}$.  Suppose to the contrary that some component of $G_i$ contains a vertex $s\in L_j$ and a vertex $t\in L_{j'}$ for some integers $j$ and $j'$ with $j'\ge j+4$.  Then any path $P$ from $s$ to $t$ in $G_i$ contains an edge $vw$ with $v\in L_a$, $w\in L_{a+1}$ for some $a\equiv 0\pmod 4$ and also contains an edge $xy$ with $x\in L_b$ and $y\in L_{b+1}$ for some $b\equiv 2\pmod 4$.  The edge $vw$ implies that $P$ is contained in $G_1$ but the edge $xy$ implies that $P$ is contained in $G_2$, a contradiction.

  Therefore, for each $i\in\{1,2\}$, each component of $G_i$ is a subgraph of $G[L_j\cup\cdots\cup L_{j+3}]$ for some $j\in\{1,\ldots,h-3\}$.  Since $G$ is $(4,c)$-good, this implies that $\odd(G_i)\le c$.  Therefore, by \cref{partition}, $\odd(G)\le \odd(G_1)+\odd(G_2)\le 2c$.
\end{proof}

With \cref{four_good} in hand, we can now easily prove \cref{layered_treewidth_result}, but first we should define all the terms used in \cref{layered_treewidth_result}.

A \emph{tree decomposition} $\mathcal{T}:=(B_x:x\in V(T))$ of a graph $G$ is a sequence of subsets of $V(G)$ called \emph{bags} that are indexed by the nodes of a tree $T$ and such that,
\begin{inparaenum}[(i)]
  \item for each vertex $v$ of $G$, $T[x\in V(T):v\in B_x]$ is a connected graph with at least one vertex; and
  \item for each edge $vw$ of $G$, ther exists a node $x$ of $T$ such that $\{v,w\}\subseteq B_x$.
\end{inparaenum}
The \emph{width} of a tree decomposition is equal to the size of its largest bag, minus $1$.  The \emph{treewidth} of $G$ is the minimum width of any tree decomposition of $G$.

\todo[inline]{
Chun-Hung suggests the following changes.

24. In the paragraph that defines the layered treewidth: I would replace
the entire paragraph by "The {\it layered treewidth} of a graph $G$ is
$\max\{|V \cap B|: V \in {\mathcal L}, B \in {\mathcal T}\}$ over all
layerings ${\mathcal L}$ of $G$ and tree-decompositions ${\mathcal T}$
of $G$."

25. In the proof of Theorem 2: Replace the first sentence by "Let
${\mathcal L}:=(V_1,V_2,...,V_h)$ be a layering of $G$ and ${\mathcal
T}:=(B_x: x \in V(T))$ be a tree-decomposition of $G$ such that
$\max\{|V_j \cap B_x|: j \in \{1,2,...,h\}, x \in V(T)\} \leq t$. Then
..."
}

A \emph{layered tree decomposition} of $G$ is a pair $(\mathcal{L},\mathcal{T})$ where $\mathcal{L}:=V_0,\ldots,V_h$ is a layering of $G$ and $\mathcal{T}:=(B_x:x\in V(T))$ is a tree decomposition of $G$.  The \emph{layered width} of $(\mathcal{L},\mathcal{T})$ is $\max\{|L_j\cap B_x|:j,x\in\{0,\ldots,h\}\times V(T)\}$.  The \emph{layered treewidth} of $G$ is the minimum layered width of any layered tree decomposition of $G$.

\begin{proof}[Proof of \cref{layered_treewidth_result}]
  Let $\mathcal{L}:=L_1,\ldots,L_h$ and $\mathcal{T}:=(B_x:x\in v(T))$ be a layering and tree decomposition of $G$ such that the layered width of $(\mathcal{L},\mathcal{T})$ is at most $t$.  Then, for any $i\in\{1,\ldots,h-3\}$, $(B_x\cap(L_i\cup\cdots\cup L_{i+3}):x\in V(T))$ is a tree decomposition of $G[L_i\cup\cdots\cup L_{i+3}]$ whose largest bag has size at most $4t$.  Therefore, the treewidth of $G[L_i\cup\cdots\cup L_{i+3}]$ is at most $4t-1$.  The class of graphs of treewidth $4t-1$ is a $(4t-1)$-degenerate minor-closed family of graphs.  Therefore, by \cref{degenerate}, $\odd(G[L_i\cup\cdots\cup L_{i+3}])\le 8t-1$.  Therefore $G$ is $(4,8t-1)$-good so, by \cref{four_good}, $\odd(G)\le 16t-2$.
\end{proof}

\subsection{A Digression}

\cref{four_good} shows that if $G$ is $(4,c)$-good then $\odd(G)$ is upper bounded by a function of $c$.  This leads to the following question:  ``What is the minimum value of $\ell$ such that any $(\ell,c)$-good graph $G$ has $\odd(G)$ upper bounded by a function of $c$?''  As it turns out, the answer is $2$, as the following results show.

\begin{lem}\label{three_good}
  If a graph $G$ has a $(3,c)$-good layering then $\odd(G)\le 3c$.
\end{lem}

\begin{proof}
  This proof is similar to the proof of \cref{four_good}, but partitions $G$ into three graphs $G_0$, $G_1$, and $G_2$.  We may assume that $G$ has no isolated vertices. Let $\mathcal{L}:=L_1,\ldots,L_h$ be a $(3,c)$-good layering of $G$. For each $i\in\{0,1,2\}$, let $X_i:=\cup\{L_j:j\equiv i\pmod 3\}$.

  Let $I_{01}$ be the set of isolated vertices in $G[X_0\cup X_1]$ and define $Y_2:=X_2\cup I_{01}$.  Let $I_2$ be the set of isolated vertices in $G[Y_2]$, let $V_2:=Y_2\setminus I_2$ and let $\overline{V}_2:=V(G)\setminus V_2$.  Observe that neither $G[V_2]$ nor $G[\overline{V}_2]$ has isolated vertices.

  Let $I_1$ be the set of isolated vertices in $G[X_1\setminus V_2]$ and observe that every vertex $v$ in $I_1$ is adjacent, in $G$ to some vertex $w$ in $X_0\setminus V_2$.  Let $Y_0:=(X_0\setminus V_2)\cup I_1$ and let $I_0$ be the set of isolated vertices in $G[Y_0]$.  Let $V_0:=Y_0\setminus I_0$ and let $V_1:=V(G)\setminus(V_0\cup V_2)$.  It is straightforward to verify that neither $G[V_0]$ nor $G[V_1]$ has isolated vertices.

  We claim that, for reach $i\in\{0,1,2\}$, each component of $G[V_i]$ contains vertices from at most $3$ consecutive layers of $\mathcal{L}$.  Indeed, otherwise such a component contains an edge $vw$ with $v\in L_a$ and $w\in L_{a+1}$ where $a\not\equiv i\pmod 3$ and $a+1\not\equiv i\pmod 3$.  If $i=2$, this is not possible because neither $v$ nor $w$ is an isolated vertex of $G[X_0\cup X_1]$.  If $i=0$ or $i = 1$ this is not possible because it would imply that one of $v$ or $w$ is in $V_2$.

  Therefore, since $G$ is $(3,c)$ good, $\odd(G[V_i])\le c$ for each $i\in\{0,1,2\}$ so, by \cref{partition}, $\odd(G)\le 3c$.
\end{proof}

\begin{lem}
  If a graph $G$ has a layering that is $(2,c)$-good then $\odd(G)\le c^2$.
\end{lem}

\begin{proof}
  Let $\mathcal{L}:=L_1,\ldots,L_h$ be a $(2,c)$-good layering of $G$.  Let $G_0$ be the (not necessarily induced) subgraph of $G$ that contains each edge $vw$ of $G$ such that $v\in L_i$ for some even integer $i$ and $w\in L_i\cup L_{i+1}$.  Similarly, let $G_1$ be the subgraph of $G$ that contains each edge $vw$ of $G$ such that $v\in L_i$ for some odd integer $i$ and $w\in L_i\cup L_{i+1}$.  Observe that $E(G_0)$ and $E(G_1)$ form a partition of $E(G)$.  Therefore, for any non-isolated vertex $v$ of $G$, $v$ is non-isolated in at least one of $G_0$ or $G_1$.  Furthermore, for each $i\in\{0,1\}$, each component of $G_i$ contains vertices from at most $2$ consecutive layers of $\mathcal{L}$.  Since $\mathcal{L}$ is $2$-good, there exists a proper odd colouring $\varphi_i:V(G_i)\to\{1,\ldots,c\}$ of $G_i$.  For each $j\in\{1,\ldots,h\}$ and each $v\in L_j$, define $\varphi_2(v):= j\bmod 3$ and define $\varphi(v):=(\varphi_2(v),\varphi_1(v),\varphi_0(v))$.

  Then $\varphi$ is a colouring of $G$ that uses at most $3c^2$ colours.  To see that $\varphi$ is a proper colouring of $G$ recall that each edge $vw$ of $G$ is an edge of $G_i$ for some $i\in\{0,1\}$, so $\varphi_i(v)\neq\varphi_i(w)$ which implies that $\varphi(v)\neq\varphi(w)$.  To see that $\varphi$ is an odd colouring of $G$, recall that each vertex $v$ of $G$ is non-isolated in $G_i$ for some $i\in\{0,1\}$.  Without loss of generality, assume that $v$ is non-isolated in $G_0$, so some colour $\alpha$ appears an odd number of times in the multiset $\{\{\varphi_i(w):w\in N_{G_0}(v)\}\}$. Therefore, some colour $(\beta,\alpha)$ appears an odd number of times in the multiset $\{\{(\varphi_1(w),\varphi_{0}(w)):w\in N_{G_0}(v)\}\}$.

  Now, the colour $(\beta,\alpha)$ may also appear arbitrarily many times in the multiset $\{\{(\varphi_1(w'),\varphi_{0}(w')):w\in N_{G_{1}}(v)\}\}$.  However, for each $w'\in N_{G_{1}}(v)$ and each $w\in N_{G_0}(v)$, $\varphi_2(w')\neq \varphi_2(w)$.  Therefore, there exists a colour $\gamma\in\{0,1,2\}$ such that $(\gamma,\beta,\alpha)$ appears an odd number of times in the multiset $\{\{(\varphi_2(w),\varphi_1(w),\varphi_{0}(w)):w\in N_{G}(v)\}\}$.  Therefore $\gamma$ is an odd colouring of $G$.
\end{proof}

\begin{obs}
  For each $q\in\N$, there exists a graph $G$ with a $(1,1)$-good layering such that $\odd(G)\ge q$.
\end{obs}

\begin{proof}
  Let $G_0$ be any graph whose chromatic number is at least $q$ and let $G$ be obtained from $G_0$ by subdividing each each exactly once.  Then $G$ is bipartite and therefore has a layering $\mathcal{L}:=L_1,L_2$ in which $G[L_1]$ and $G[L_2]$ each have no edges, $\odd(G[L_1])=\odd(G[L_2])=1$, so $\mathcal{L}$ is a $(1,1)$-good layering of $G$.  Furthermore, in any odd colouring $\varphi$ of $G$, each subdivision vertex $x$ must be adjacent to two vertices $v$ and $w$ of different colours.  But for any edge $vw$ of $G_0$ there is a subdivision vertex $x$ with $N_G(x)=\{v,w\}$, so $\varphi(v)\neq\varphi(w)$.  Therefore $\varphi$ is a proper colouring of $G_0$, so $\varphi$ uses at least $q$ colours.
\end{proof}

\section{Product Structure (Proof of \cref{product_structure_result})}
\label{product_structure}

For two graphs $A$ and $B$, the \emph{strong product} $A\boxtimes B$ of $A$ and $B$ is the graph with vertex set $V(A\boxtimes B):=V(A)\times V(B)$ and that contains an edge with endpoints $(x_1,y_1)$ and $(x_2,y_2)$ if and only if
\begin{inparaenum}[(i)]
  \item $x_1x_2\in E(A)$ and $y_1=y_2$;
  \item $x_1=x_2$ and $y_1y_2\in E(B)$; or
  \item $x_1x_2\in E(A)$ and $y_1y_2\in E(B)$.
\end{inparaenum}
A \emph{$t$-tree} $H$ is a graph that is either a clique on $t+1$ vertices or a graph that contains a vertex $v$ of degree $t$ whose neighbours form a clique and such that $H-\{v\}$ is a $t$-tree.

Our proof of \cref{product_structure_result} is inspired by the proof \cref{degenerate} by \citet{cranston.lafferty.ea:note}, which is an inductive proof that involves contracting an edge $wv$ incident to a vertex $v$ of degree at most $d$.  The contraction of this edge (as opposed to the deletion of $v$) is crucial to ensuring that $N_G(v)$ has a colour (namely $\varphi(w)$) that appears an odd number of times.  However, since the class of graphs with product structure is not minor-closed we can not use edge contractions. Instead, we use vertex deletion along with a condition labelled (\ding{96}) to achieve a similar effect.

\begin{proof}
  Let $L:V(G)\to\binom{\N}{8t+4}$ be an $(8t+4)$-list assignment of $G$, so our job is to construct a proper odd $L$-colouring $\varphi$ of $G$.  Let $y_1,\ldots,y_h$ be the vertices of $P$, in order.  To avoid a boring edge case, we extend $P$ by one vertex in each direction, so that the vertices $y_0$ and $y_{h+1}$ are defined.

  Let $x_1,\ldots,x_r$ be the vertices of $H$ ordered so that $x_1,\ldots,x_{t}$ is a clique and, for each $i\in\{t+1,\ldots,r\}$, the vertices in $C_{x_i}:=N_G(x_i)\cap\{x_1,\ldots,x_{i-1}\}$ form a clique of order $t$.  We make crucial use of the following well-known property of $t$-trees:\footnote{(\ding{74}) follows from the fact that $S:=C_{x_i}\cup\{x_i\}$ separates $\{x_1,\ldots,x_i\}\setminus S$ from $x_{j}$.  In the language of rooted tree-decompositions, the node whose bag contains $S$ is an ancestor of all nodes whose bags contains $x_{j}$.}
  \begin{compactitem}[(\ding{74})]
    \item If $1\le i \le j\le r$ and $x_i \in C_{x_{j}}$ then $C_{x_i}\cup\{x_i\}\supseteq C_{x_{j}}\cap\{x_1,\ldots,x_{i}\}$.
  \end{compactitem}

  With this setup out of the way, we can proceed with the proof, which is by induction on the number of vertices of $G$.  We will prove the following stronger statement:  There exists a proper odd $L$-colouring $\varphi$ of $G$ that also satisifies the following condition:
  \begin{compactitem}[(\ding{96})]
    \item For each $(x_i,y_j)\in V(H\boxtimes P)$, define
    \[
      C_{(x_i,y_j)}:=V(G)\cap \left(\left(C_{x_i}\times\{y_{j-1}, y_{j},y_{j+1}\}\right)\cup\{(x_i,y_i),(x_i,y_{j-1})\}\right)
      \enspace .
    \]
    Then, for each $v\in V(H\boxtimes P)$ the vertices in $C_v$ receive distinct colours, i.e., $\varphi(u)\neq \varphi(w)$ for each distinct $u,w\in C_v$.
  \end{compactitem}
  Note that, for any edge $vw$ of $G$, $v,w \in C_w$ or $v,w\in C_v$.  Therefore, (\ding{96}) implies that the colouring $\varphi$ is a proper colouring of $G$.

  The base case, in which $G$ has no vertices, is trivial. Therefore, we may assume that $G$ has at least one vertex.  Let $(i,j)\in\{1,\ldots,r\}\times\{1,\ldots,h\}$ be the lexicographically largest pair such that $v:=(x_i,y_j)\in V(G)$.  Observe that $|N_G(v)|\le 3t+1$ since, by the maximality of $(i,j)$,  $N_G(v)\subseteq C_v\setminus\{v\}$.

  Consider the graph $G':= G-\{v\}$. Since $G'$ is a subgraph of $H\boxtimes P$ with fewer vertices than $G$, the inductive hypothesis implies that there exists an odd colouring $\varphi$ of $G'$ that satisfies (\ding{96}).  We will now extend $\varphi$ to a colouring of $G$ by listing colours that we may not choose for $\varphi(v)$:
  \begin{itemize}
    \item To guarantee that $\varphi$ satisfies (\ding{96}), observe that assigning a colour to $v$ can only violate (\ding{96}) if it does so for some vertex $(x_{i'},y_{j'})\in V(H\boxtimes P)$ with $v\in C_{(x_{i'},y_{j'})}$.  By (\ding{74}) and the maximality of $(i,j)$, if $v\in C_{(x_{i'},y_{j'})}$ then $|j-j'|\le 1$ and $C_{x_{i}}\cup\{x_i\}\supseteq C_{x_{i'}}$.  Therefore,
    \[
        C_{(x_{i'},y_{j'})}\setminus\{v\}\subseteq
          \left(C_{x_i} \times \{y_{j-2},y_{j-1},y_j,y_{j+1},y_{j+2}\}\right)
          \cup \{(x_i,y_{j-2}),(x_i,y_{j-1})\} =: R \enspace .
    \]
    Therefore, in order to satisfy (\ding{96}) it is sufficient to choose $\varphi(v)$ so that $\varphi(v)\neq\varphi(w)$ for each $w\in R$.  Let $X:=\{\varphi(w): w\in R\}$ and observe that $|X|\le |R| \le 5t+2$.

    Furthermore, if $v$ is not an isolated vertex of $G$ then (\ding{96}) ensures that some colour occurs exactly once in $N_G(v)$. Therefore, to ensure that $\varphi$ is an odd colouring of $G$, we need only choose some $\varphi(v)\not\in X$ in such a way that some colour appears an odd number of times in $N_G(w)$ for each $w\in N_G(v)$, which is what we do next.

    \item To guarantee that $\varphi$ is an odd colouring of $G$, consider each vertex $w\in N_{G}(v)$.  If there is exactly one colour $\alpha$ that occurs an odd number of times in $N_{G'}(w)$, then define $Y_{w} := \{\alpha\}$; otherwise define $Y_{w}:=\emptyset$. Now let $Y:=\bigcup_{w\in N_{G}(v)} Y_{w}$ and observe that $|Y|\le |N_G(v)|\le 3t+1$.
    If we choose $\varphi(v)\not\in Y$ then , for each $w\in N_{G}(v)$ the following holds:
    \begin{itemize}
      \item If $Y_{w}=\{\alpha\}$ then $\varphi(v)\neq\alpha$. Therefore, the colour $\alpha$ appears an odd number of times in $N_{G}(w)$ since it appears an odd number of times in $N_{G'}(w)=N_G(w)\setminus\{v\}$.
      \item If $Y_{w}=\emptyset$ then either:
      \begin{itemize}
        \item No colour appears an odd number of times in $N_{G'}(w)$.  Therefore the colour $\varphi(v)$ appears an even number of times in $N_{G'}(w)$, so $\varphi(v)$ appears an odd number of times in $N_G(w)=N_{G'}(w)\cup\{v\}$.
        \item At least two colours $\alpha$ and $\beta$ each appear an odd number of times in $N_{G'}(w)$.  Therefore each colour in $\{\alpha,\beta\}\setminus\{\varphi(v)\}$ appears an odd number of times in $N_{G}(w)$.  In particular, at least one of $\alpha$ or $\beta$ appears an odd number of times in $N_G(w)$.
      \end{itemize}
    \end{itemize}
  \end{itemize}
  Therefore, by choosing $\varphi(v)\not\in X\cup Y$ we obtain an odd colouring of $G$ that satisifies (\ding{96}).  Since $|X\cup Y|\le |X|+|Y|\le 8t+3 < |L(v)|$, there exists some $\varphi(v)\in L(v)\setminus(X\cup Y)$ that completes the colouring of $G$.
\end{proof}


\subsection{Remarks}

%
% One might hope that \cref{product_structure_result} could be generalized to the setting in which $H$ belongs to some $t$-degenerate minor-closed family of graphs.  However, bounding the size of the set $X$ required to maintain (\ding{96}) when choosing $\varphi(v)$ relies critically on (\ding{74}), which is a property of graphs of treewidth $t$ that is not true for all $t$-degenerate minor-closed graph families.

% \paragraph{Subgraphs of $H\boxtimes P\boxtimes K_\ell$}

A number of product structure theorems characterize graphs as subgraphs of $H\boxtimes P\boxtimes K_{\ell}$ where $H$ is a $t$-tree, $P$ is a path, and $K_\ell$ is a complete graph of order $\ell$.  Since $H\boxtimes P\boxtimes K_{\ell}$ is isomorphic to $H\boxtimes K_\ell\boxtimes P$ and $H\boxtimes K_\ell$ is a $(\ell(t+1)-1)$-tree, \cref{product_structure_result} immediately implies that these graphs have odd colourings using $8\ell t+8\ell-4$ colours.

It is possible to improve this slightly by redoing the proof \cref{product_structure_result}.  In this case, the vertex $v:=(x_i,y_j,z_k)$ that is removed is also chose to maximize $(i,j)$, with ties broken arbitrarily. Then one finds that the sizes of the colour sets $X$ and $Y$ that must be avoided when choosing $\varphi(v)$ are bounded by $|X|\le 5\ell t + 3\ell-1$ and $|Y|\le 3\ell t + 2\ell -1$ so that $|X\cup Y|\le 8\ell t + 5\ell -2$. This gives the following variant of \cref{product_structure_result}:

\begin{thm}\label{product_structure_result_kl}
  If $G$ is a subgraph of $H\boxtimes P\boxtimes K$, where $H$ is a $t$-tree, $P$ is a path, and $K$ is a clique of order $\ell$, then $\oddc(G)\le 8\ell t + 5\ell -1$.
\end{thm}

% \paragraph{Subgraphs of $H\boxtimes I$}

Perhaps a more interesting generalization comes by replacing $P$ by some graph $I$ of maximum-degree $\Delta$.  Again, one can follow the same general strategy used in the proof of \cref{product_structure_result}, with the following changes.
\begin{compactitem}
  \item The vertices $y_1,\ldots,y_h$ are the vertices of $I$ is no particular order.
  \item The set $C_{(x_i,y_i)}$ is defined as
\[
    C_{(x_i,y_i)}:=(\{x_i\}\cup C_{x_i})\boxtimes (\{y_j\}\cup N_I(y_j)) \enspace .
\]
  \item $|X\cup Y|$ is bounded as follows: $|Y|\le|N_G(v)|\le |C_v\setminus\{v\}| \le (t+1)(\Delta+1)-1$.  The set $R$ used to define $X$ is given by $R:=(\{x_i\}\cup C_{x_i})\times N^2_I(y_i)$, where $N^2_I(y_i)$ denotes the set of at most $\Delta^2+1$ vertices in $I$ of distance at most $2$ from $y_i$.  Then $|X|\le|R\setminus\{v\}|\le (t+1)(\Delta^2+1)-1$.  Therefore $|X\cup Y|\le |X|+|Y|\le (\Delta^2+\Delta)(t+1)+2t$.
\end{compactitem}
These changes prove the following variation of \cref{product_structure_result}:

\begin{thm}\label{product_structure_result_delta}
  If $G$ is a subgraph of $H\boxtimes I$ where $H$ is a $t$-tree and $I$ is a graph of maximum-degree $\Delta$, then $\oddc(G)\le(\Delta^2+\Delta)(t+1)+2t+1$.
\end{thm}



%
% We conclude by noting that the proof of \cref{product_structure_result} can easily be modified to prove the following refinement:
%
% \todo[inline]{We have to rethink this next part now.}
% \begin{thm}\label{product_structure_result2}
%   Let $H$ be a $t$-tree, let $B$ be a $d$-degenerate\footnote{A graph $G$ is $d$-degenerate if every subgraph of $G$ contains a vertex of degree at most $d$.} graph of maximum-degree $\Delta$, and let $G$ be a subgraph of $H\boxtimes B$. Then $G$ has a proper odd colouring $\varphi:V(H)\to\{1,\ldots,2((\Delta+1)t + d)+1\}$.
% \end{thm}
%
% The proof of \cref{product_structure_result2} is almost identical to the proof of \cref{product_structure_result} except for the following modifications:  Let $y_1,\ldots,y_h$ be an ordering of $V(B)$ for which $|N_B(y_j)\cap\{y_1,\ldots,y_{j}\}|\le d$ for each $j\in\{1,\ldots,h\}$ and let $C_{y_j}:=N_B(y_j)\cap\{y_1,\ldots,y_{j}\}$.  Then the definition of $C_{x_i,y_j}$ becomes
% \[
%    C_{x_i,y_j} := V(G)\cap \left( (C_{x_i}\times (\{y_j\}\cup N_B(y_j)))
%        \cup \{(x_i,y): y \in C_{y_j}\}  \right) \enspace .
% \]
% As before the vertex $v:=(x_i,y_j)\in V(G)$ is chosen so that $(i,j)$ is lexicographically maximum.  With this choice of $v$,  $N_G(v)\subseteq C_v$, so $|N_G(v)|\le (\Delta+1)t+d$. Since $|X|\le |C_v|$ and $|Y|\le |N_G(v)|$, this means there are at most $2((\Delta+1)t+d)$ colours that must be avoided when choosing $\varphi(v)$, so $2((\Delta+1)t+d)+1$ colours are sufficient. \Cref{product_structure_result} is a consequence of \cref{product_structure_result2} because a path $P$ is a graph with maximum degree $\Delta=2$ and degeneracy $d=1$.

\bibliographystyle{plainurlnat}
\bibliography{odd}


\end{document}
